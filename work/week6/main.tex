\documentclass[11pt]{scrartcl}
\usepackage[T1]{fontenc}
\usepackage[utf8]{inputenc}

\title{Computer Systems Week 6 Lab}
\author{Daniel Coady (102084174)}
\date{19/09/2019}

\usepackage{graphicx}

\begin{document}

\maketitle

\section*{Assignment Progress}
\subsection*{What I have so far}
At this point in time, I have managed to get a total of 3 stages done, those being the
play/pause, volume control, and track skipping stages. Right now I feel like I'm in a good
place since (especially on the last stage) these have been the hardest things to plan
out and implement. The last two stages of this assignment entail turning on and off the
circuit, which should be fairly trivial to implement. I already have some rough ideas
of how to control this for each individual circuit, mostly coming down to use of and
gates to control the flow of data and using registers to store state when turned off and
restore state when turned on.

\subsection*{Challenges}
Stages 1 and 2 were quite easy to plan and implement based off the knowledge gained from
the lectures, but stage 3 wasn't as easy. There were a few reasons why this was the case:
\begin{itemize}
    \item Finding the right counter design wasn't as easy as initially thought
    \item Trying to choose and justify why the circuit should be synchronous or not
    \item Bi-directional counting from 0-99 is easy enough on it's own, but displaying it means we can't use a 7 bit counter
\end{itemize}
I ultimately was able to solve the issue, however it took me 2 weeks and 7 iterations to
get it right. My circuit for the track skipping ended up containing two synchronous
4 bit bi-directional MOD 10 counters, with one feeding into the other so that each counter
can act as a "column" in our more familiar decimal system. The biggest trapping for me was
ensuring that both counters could catch and correct illegal states (10 and 15 in this
case) appropriately. The reason why this was an issue is because the state of 10 (1010) is
actuallly "contained" within the state of 15 (1111) so sometimes you would get a state of
15 tripping the check for if it was at 10. My solution to this was to give each check a
control bit in the form of the next and previous controls. This works because 10 will only
ever appear as a state if we are counting up, while 15 would appear only if we were
counting down.

\subsection*{Where to from here?}
From here I have two stages left to complete. My current plan for these is to modify the
existing circuits to handle the turning on and off, but then have external circuitry
handle the storing of data into the registers. I'm still unsure of how well this will work,
but for now this a start until I get to experimenting with the different approaches that
I can take with these final tasks.

\end{document}
