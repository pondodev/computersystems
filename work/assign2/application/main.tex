\documentclass[11pt]{scrartcl}
\usepackage[T1]{fontenc}
\usepackage[utf8]{inputenc}

\title{COS10004 Computer Systems Assignment 2 Part B}
\author{Daniel Coady (102084174) -- 12:30 Wednesday}
\date{26/10/2019}

\usepackage{graphicx}
\usepackage{fancyhdr}
\pagestyle{fancy}
\lhead{Daniel Coady (102084174)}
\rhead{COS10004 Computer Systems -- Assignment 2 Part B}

\begin{document}

\maketitle

\pagebreak

\section*{Summary}
This report covers the custom application that I have created as part of
part B in assignment 2. The application is an implementation of fizzbuzz
in ARMv7 assembly that runs under an operating system, which in this case
was tested on Raspbian. Fizzbuzz is a children's game where you count from
1 to n with the following rules:
\begin{itemize}
    \item If the number is divisible by 3, you must say fizz
    \item If the number is divisible by 5, you must say buzz
    \item If the number is divisible by 3 and 5, you must say fizzbuzz
\end{itemize}
For the sake of simplicity since this would be out of scope otherwise,
I have also decided to simply print out "number" instead of the number
itself.

\section*{Example Output}
The progam when run will have an output similar to this:
\begin{verbatim}
number
number
fizz
number
buzz
\end{verbatim}

\section*{Implementation}
I have approached this task with the following ideas:
\begin{itemize}
    \item We will loop from 1 to 30
    \item We will store our "state" in r1
    \item The states will range from 0-3
    \begin{itemize}
        \item 0 is a number
        \item 1 is fizz
        \item 2 is buzz
        \item 3 is fizzbuzz
    \end{itemize}
    \item States will dictate what it is that we print to screen
\end{itemize}
With this in mind, we perform the following actions on every loop through:

\subsection*{Checking State}
First we check if the number is divisible by the given number. To achieve
this I have created the following function:
\begin{verbatim}
// parameters:
// r0 = number
// r1 = divisor
//
// return values:
// r3 = true (1) false (0)
isdivisible:
  push {r0, r2}
  udiv r2, r0, r1 // store division result in r2
  mul r2, r1      // multiply result by divisor
  mov r3, #0      // store a false return value
  cmp r2, r0      // check if r2 == r0
  addeq r3, #1    // add one to our return value if they are equal
  pop {r0, r2}
  bx lr
\end{verbatim}
This function is essentially just an implementation of the modulus operator
in other languages. The basic idea of it is:
\begin{itemize}
    \item Get the number and the divisor
    \item Divide the number by the divisor
    \item Multiply the result by the divisor
    \item Compare the number against the result
    \begin{itemize}
        \item If equal, it is divisible
        \item If not equal, it is not divisible
    \end{itemize}
\end{itemize}
This function forms the backbone of the application, being used in both
functions checkfizz and checkbuzz. Both checkfizz and checkbuzz do as
their names imply.
\begin{verbatim}
// parameters:
// r0 = number
// r1 = current result
//
// return values:
// r1 = result
checkfizz:
  push {lr}
  push {r0, r1}
  mov r1, #3
  bl isdivisible // check if r0 is divisible by 3
  pop {r0, r1}
  cmp r3, #1
  addeq r1, #1   // add 1 to result if is divisible
  pop {lr}
  bx lr
\end{verbatim}
Notice that if it does end up passing the test that we add 1 to r1. This
is the aforementioned state that we store. The state starts off at 0 and
with each passing test we add more to it. So if we pass the fizz test then
we add 1 to it and if we pass the buzz test we add 2. With this system we
are able to accurately track what we should be printing to the screen and
even make easier changes to it in the future if we want to.

\subsection*{Printing State}
Just like when checking state, when printing state we have a single function
that forms the backbone of this functionality:
\begin{verbatim}
print:
  mov r7, #4  // sys_write
  mov r0, #1  // stdout
  swi 0
  bx lr
\end{verbatim}
This is a significantly simpler function, simply setting the appropriate
registers to the correct values so that we are able to write to stdout.
This function does not work on it's own however and must be called by one
of the other functions up one level of abstraction from it:
\begin{verbatim}
sayfizz:
  push {lr}
  ldr r1, =fizz
  mov r2, #5 // length of word
  bl print
  pop {lr}
  bx lr
\end{verbatim}
Here we set the word specific things such as the ASCII array to print out
and the length of the ASCII array. From here we can call the print function
which will then print our desired word to the shell.

\end{document}
