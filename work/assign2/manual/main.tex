\documentclass[11pt]{scrartcl}
\usepackage[T1]{fontenc}
\usepackage[utf8]{inputenc}

\title{COS10004 Computer Systems Assignment 2}
\author{Daniel Coady (102084174) -- 12:30 Wednesday}
\date{14/10/2019}

\usepackage{graphicx}
\usepackage{fancyhdr}
\pagestyle{fancy}
\lhead{Daniel Coady (102084174)}
\rhead{COS10004 Computer Systems -- Assignment 2}

\begin{document}

\maketitle

\pagebreak

\section{mov}
\subsection{Syntax}
\begin{verbatim}
mov x, y
\end{verbatim}
Where:
\begin{itemize}
    \item x is the destination
    \item y is the value
\end{itemize}
\subsection{Description}
Used to move a value into a register. Note that values must have 24
consecutive zeroes in it's binary notation.
\subsection{Example}
\begin{verbatim}
mov r0, $3F0000 ; valid
mov r0, $003F00 ; valid
mov r0, $00003F ; valid
mov r0, $300F00 ; invalid
\end{verbatim}

\section{orr}
\subsection{Syntax}
\begin{verbatim}
orr x, y
\end{verbatim}
Where:
\begin{itemize}
    \item x is value 1 and the destination
    \item y is value 2
\end{itemize}
\subsection{Description}
Performs a bitwise OR operation on x and y, storing the result in x.
\subsection{Example}
\begin{verbatim}
mov r0, $10 ; r0 has 0x10
orr r0, $01 ; r0 has 0x11
\end{verbatim}

\section{ldr}
\subsection{Syntax}
\subsection{Description}
\subsection{Example}

\section{ldrd}
\subsection{Syntax}
\subsection{Description}
\subsection{Example}

\section{str}
\subsection{Syntax}
\subsection{Description}
\subsection{Example}

\section{add}
\subsection{Syntax}
\subsection{Description}
\subsection{Example}

\section{sub}
\subsection{Syntax}
\subsection{Description}
\subsection{Example}

\section{b}
\subsection{Syntax}
\subsection{Description}
\subsection{Example}

\section{push}
\subsection{Syntax}
\subsection{Description}
\subsection{Example}

\section{pop}
\subsection{Syntax}
\subsection{Description}
\subsection{Example}

\end{document}
