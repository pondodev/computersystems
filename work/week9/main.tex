\documentclass[11pt]{scrartcl}
\usepackage[T1]{fontenc}
\usepackage[utf8]{inputenc}

\title{Computer Systems Week 9 Lab}
\author{Daniel Coady (102084174)}
\date{09/10/2019}

\usepackage{graphicx}

\begin{document}

\maketitle

\section*{How the Code Works}

\subsection*{kernel7.asm}
In this source file we simply get the factorial of 4 and then flash the LED that many
times.

\subsection*{TIMER.asm}
In this source file we simply wait the amount of time passed through to the TIMER function.

\subsection*{factorialj.asm}
In this source file we recursively calculate the factorial of the number passed through
to the FACTORIAL function.

\subsection*{Program Input}
The input to the program is in r1. This value is then used to get a factorial which is
used as the amount of flashes to be displayed by the LED connected to GPIO18.

\pagebreak

\section*{My Modified Code}

\subsection*{kernel7.asm}
\begin{verbatim}
BASE = $3F000000

;Calculate
mov r1,#4 ; input
mov sp,$1000 ; make room on the stack
mov r0,r1  
bl FACTORIAL
mov r7,r0 ; store answer
bl init ; initialise GPIO base address and GPIO18 for output

mov r1, r7 ; store amount of flashes
bl flash ; flash the LED
wait:
b wait

include "TIMER.asm"
include "factorialj.asm"
include "INIT.asm"
include "FLASH.asm"
\end{verbatim}

\subsection*{TIMER.asm}
\begin{verbatim}
; TIMER
; no args
TIMER_OFFSET = $3000
DELAY = $20000
TIMER:
  mov r0, BASE
  orr r0, TIMER_OFFSET ; timer address
  mov r1, DELAY ; store delay
  ldrd r2, r3, [r0, #4] ; store the start time
  mov r4, r2 ; store least significant 32-bits of start time in r4
  
  timerdelay:
    ldrd r2, r3, [r0, #4] ; store the current time
    sub r5, r2, r4 ; r5 = current time - start time
    cmp r5, r1
  bls timerdelay ; goto timerdelay if r5 <= delay
bx lr
\end{verbatim}

\subsection*{factorialj.asm}
\begin{verbatim}
FACTORIAL:
sub r1,r1,#1
cmp r1,#1
beq EXIT
mul r0,r0,r1
push {r1,lr}
; push onto the stack without changing the stack pointer
bl FACTORIAL	  ; call FACTORIAL
EXIT:
pop {r1,lr}  ; pop off the stack
bx lr	; RETURN
\end{verbatim}

\subsection*{INIT.asm}
\begin{verbatim}
; init function -- sets up GPIO18 for writing and retrieves GPIO base address
; no arguments
; returns BASE + GPIO_OFFSET
GPIO_OFFSET = $200000
init:
  mov r0,BASE
  orr r0,GPIO_OFFSET
  mov r1,#1
  lsl r1,#24
  str r1,[r0,#4] ; set GPIO18 to output
bx lr
\end{verbatim}

\subsection*{FLASH.asm}
\begin{verbatim}
; flash function -- will flash the LED on GPIO18 a set amount of times
; r0 = GPIO base address
; r1 = num of times to flash
; returns nothing
flash:
  loop$:
    mov r2,#1
    lsl r2,#18
    str r2,[r0,#28]  ; turn LED on
    mov r3,$0F0000

    push {lr, r0, r1, r2, r3, r4, r5}
    bl TIMER
    pop {lr, r0, r1, r2, r3, r4, r5}
    
    mov r2,#1
    lsl r2,#18
    str r2,[r0,#40]  ; turn LED off
    mov r3,$0F0000

    push {lr, r0, r1, r2, r3, r4, r5}
    bl TIMER
    pop {lr, r0, r1, r2, r3, r4, r5}

  sub r1,#1
  cmp r1,#0
  bne  loop$  ; end of outer loop. Runs r1 times
bx lr
\end{verbatim}

\end{document}