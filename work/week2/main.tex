\documentclass[11pt]{scrartcl}
\usepackage[T1]{fontenc}
\usepackage[utf8]{inputenc}

\title{Computer Systems Week 2 Lab}
\author{Daniel Coady (102084174)}
\date{07/08/2019}

\usepackage{graphicx}
\usepackage[colorlinks]{hyperref}
\usepackage{tgadventor}

\begin{document}

\sffamily

\maketitle

% 4-bit binary adder
\begin{center}
    \begin{tabular}{c c|c}
        \multicolumn{3}{c}{4-Bit Binary Adder} \\
        \hline
        input0 & input1 & output \\
        0101 & 0000 & 00101 \\
        0101 & 0001 & 00110 \\
        0101 & 0010 & 00111 \\
        0101 & 0011 & 01000 \\
        0101 & 0100 & 01001 \\
        0101 & 0101 & 01010 \\
        0101 & 0110 & 01011 \\
        0101 & 0111 & 01100 \\
        0101 & 1000 & 01101 \\
        0101 & 1001 & 01110 \\
        0101 & 1010 & 01111 \\
        0101 & 1011 & 10000 \\
        0101 & 1100 & 10001 \\
        0101 & 1101 & 10010 \\
        0101 & 1110 & 10011 \\
        0101 & 1111 & 10100 \\
    \end{tabular}

    \includegraphics[scale=0.5]{images/4bitadder.png}
\end{center}

\pagebreak

% rs flip flop
\begin{center}
    \begin{tabular}{c c|c c}
        \multicolumn{4}{c}{RS Flip Flop} \\
        \hline
        set & reset & Qa & Qb \\
        1 & 0 & 0 & 1 \\
        1 & 1 & 0 & 0 \\
        0 & 1 & 1 & 0 \\
        1 & 1 & 0 & 0 \\
    \end{tabular}

    \includegraphics[scale=0.6]{images/rsflipflop.png}
\end{center}

When both inputs are on then both outputs turn off, which in a flip flop is an invalid state since Qb should always be the inverse of Qa.
The reason why this is an issue is because if we go from a state where both inputs are on to one where both are off, in the physical
realm we cannot predict what it's output would be since it would ultimately depend on which one is off first.
{\color{red}(double check answer to this)}

% d flip flop
\begin{center}
    \begin{tabular}{c c|c c}
        \multicolumn{4}{c}{D Flip Flop} \\
        \hline
        clk & set & Qa & Qb \\
        0 & 0 & 0 & 1 \\
        0 & 1 & 0 & 1 \\
        1 & 1 & 0 & 0 \\
        1 & 0 & 0 & 0 \\
    \end{tabular}

    \includegraphics[scale=0.6]{images/dflipflop.png}
\end{center}

The D Flip Flop works by matching the set input when the clock input is on. This allows us to synchronize it to a circuit's clock which is
useful in circuit design for time sensitive tasks where you need to be sure that certain actions are synchroized with each other. This is
used over the RS Flip Flop because it is both synchronized to a clock as well as completely safe to indeterminite states, or in other
words the output is always predictable.

\end{document}
