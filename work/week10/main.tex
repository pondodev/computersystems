\documentclass[11pt]{scrartcl}
\usepackage[T1]{fontenc}
\usepackage[utf8]{inputenc}

\title{Computer Systems Week 10 Lab}
\author{Daniel Coady (102084174)}
\date{23/10/2019}

\usepackage{graphicx}

\begin{document}

\maketitle

\section*{drawpixel}
This source file contains the function that allows us to draw pixels to
the screen, as the name suggests. The parameters are:
\begin{itemize}
  \item r0 -- The screen memory address
  \item r1 -- The x coordinate to draw to
  \item r2 -- The y coordinate to draw to
  \item r3 -- The colour to draw
\end{itemize}
r1 and r2 (our x and y values) are modified in the same way, being
prepared to indicate the correct bits to be written to in order to draw
to the right location on screen. Both values are then added to r0 (our
screen address). After this is done, we check how many bits we have per
pixel to determine what colour we can draw to the screen. Once decided,
depending on the colour bit space we are able to use we will then display
the appropriate pixel with it's assigned colour.

\section*{Code that is calling drawpixel}
\begin{verbatim}
lineloop:
   push {r0-r3}
   mov r0,r7    ;screen address
   mov r1,r4 ;x
   mov r2,r5 ;y
   mov r3,r6 ;colour
      ;assume BITS_PER_PIXEL, SCREEN_X are shared constants
       bl drawpixel
   pop {r0-r3}

;increment and test
  add r4,#1
  mov r8,SCREEN_X AND $FF00
  orr r8,SCREEN_X AND $00FF    ;640 = 0x0280
  cmp r4,r8
bls lineloop      ;branch less than or same
\end{verbatim}
The above code is used to draw a line along the top of the screen for the
entire set display's width.

\section*{Code for drawing a square to screen}
\begin{verbatim}
; drawing a square
; set starting point
mov r4, #1
mov r5, #10
squarehor: ; horizontal lines of the square
  push {r0-r3}
  mov r0, r7 ; get the screen address
  mov r1, r4 ; store x coord
  mov r2, r5 ; store y coord
  mov r3, r6 ; store colour
  bl drawpixel ; draw the pixel
  
  mov r0, r7 ; get the screen address
  add r2, #50
  bl drawpixel
  pop {r0-r3}

  add r4, #1
  cmp r4, #50
bls squarehor

; set starting point
mov r4, #1
mov r5, #10
squarever: ; vertical lines of the square
  push {r0-r3}
  mov r0, r7 ; get the screen address

  mov r1, r4 ; store x coord
  mov r2, r5 ; store y coord
  mov r3, r6 ; store colour
  bl drawpixel ; draw the pixel

  mov r0, r7 ; get the screen address
  add r1, #50
  bl drawpixel
  pop {r0-r3}

  add r5, #1
  cmp r5, #60
bls squarever
\end{verbatim}

\end{document}